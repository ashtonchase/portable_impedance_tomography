
%% cpe631.tex
%% 2017/04/29
%% by Ashton Johnson & Ian Swepston
%% see http://github.com/ashtonchase/cpe631_report/
%% for current contact information.
%%
%% This is a skeleton file demonstrating the use of IEEEtran.cls
%% (requires IEEEtran.cls version 1.8b or later) with an IEEE
%% journal paper.
%%
%% Support sites:
%% http://www.michaelshell.org/tex/ieeetran/
%% http://www.ctan.org/pkg/ieeetran
%% and
%% http://www.ieee.org/

%%*************************************************************************
%% Legal Notice:
%% This code is offered as-is without any warranty either expressed or
%% implied; without even the implied warranty of MERCHANTABILITY or
%% FITNESS FOR A PARTICULAR PURPOSE! 
%% User assumes all risk.
%% In no event shall the IEEE or any contributor to this code be liable for
%% any damages or losses, including, but not limited to, incidental,
%% consequential, or any other damages, resulting from the use or misuse
%% of any information contained here.
%%
%% All comments are the opinions of their respective authors and are not
%% necessarily endorsed by the IEEE.
%%
%% This work is distributed under the LaTeX Project Public License (LPPL)
%% ( http://www.latex-project.org/ ) version 1.3, and may be freely used,
%% distributed and modified. A copy of the LPPL, version 1.3, is included
%% in the base LaTeX documentation of all distributions of LaTeX released
%% 2003/12/01 or later.
%% Retain all contribution notices and credits.
%% ** Modified files should be clearly indicated as such, including  **
%% ** renaming them and changing author support contact information. **
%%*************************************************************************


% *** Authors should verify (and, if needed, correct) their LaTeX system  ***
% *** with the testflow diagnostic prior to trusting their LaTeX platform ***
% *** with production work. The IEEE's font choices and paper sizes can   ***
% *** trigger bugs that do not appear when using other class files.       ***                          ***
% The testflow support page is at:
% http://www.michaelshell.org/tex/testflow/



%\documentclass[journal]{IEEEtran}
%
% If IEEEtran.cls has not been installed into the LaTeX system files,
% manually specify the path to it like:
 \documentclass[journal]{IEEEtran}


% Some very useful LaTeX packages include:
% (uncomment the ones you want to load)


% *** MISC UTILITY PACKAGES ***
%
%\usepackage{ifpdf}
% Heiko Oberdiek's ifpdf.sty is very useful if you need conditional
% compilation based on whether the output is pdf or dvi.
% usage:
% \ifpdf
%   pdf code
% \else
%    dvi code
% \fi
% The latest version of ifpdf.sty can be obtained from:
% http://www.ctan.org/pkg/ifpdf
% Also, note that IEEEtran.cls V1.7 and later provides a builtin
% \ifCLASSINFOpdf conditional that works the same way.
% When switching from latex to pdflatex and vice-versa, the compiler may
% have to be run twice to clear warning/error messages.






% *** CITATION PACKAGES ***
%
\usepackage{cite}
% cite.sty was written by Donald Arseneau
% V1.6 and later of IEEEtran pre-defines the format of the cite.sty package
% \cite{} output to follow that of the IEEE. Loading the cite package will
% result in citation numbers being automatically sorted and properly
% "compressed/ranged". e.g., [1], [9], [2], [7], [5], [6] without using
% cite.sty will become [1], [2], [5]--[7], [9] using cite.sty. cite.sty's
% \cite will automatically add leading space, if needed. Use cite.sty's
% noadjust option (cite.sty V3.8 and later) if you want to turn this off
% such as if a citation ever needs to be enclosed in parenthesis.
% cite.sty is already installed on most LaTeX systems. Be sure and use
% version 5.0 (2009-03-20) and later if using hyperref.sty.
% The latest version can be obtained at:
% http://www.ctan.org/pkg/cite
% The documentation is contained in the cite.sty file itself.






% *** GRAPHICS RELATED PACKAGES ***
%
\ifCLASSINFOpdf
   \usepackage{graphicx}
  % declare the path(s) where your graphic files are
%   \graphicspath{{../pdf/}{../jpeg/}}
  % and their extensions so you won't have to specify these with
  % every instance of \includegraphics
   \DeclareGraphicsExtensions{.pdf,.jpeg,.png}
\else
  % or other class option (dvipsone, dvipdf, if not using dvips). graphicx
  % will default to the driver specified in the system graphics.cfg if no
  % driver is specified.
   \usepackage[pdftex]{graphicx}
  % declare the path(s) where your graphic files are
 %  \graphicspath{{../pdf/}}
  % and their extensions so you won't have to specify these with
  % every instance of \includegraphics
%   \DeclareGraphicsExtensions{.eps,.pdf}
\fi
% graphicx was written by David Carlisle and Sebastian Rahtz. It is
% required if you want graphics, photos, etc. graphicx.sty is already
% installed on most LaTeX systems. The latest version and documentation
% can be obtained at: 
% http://www.ctan.org/pkg/graphicx
% Another good source of documentation is "Using Imported Graphics in
% LaTeX2e" by Keith Reckdahl which can be found at:
% http://www.ctan.org/pkg/epslatex
%
% latex, and pdflatex in dvi mode, support graphics in encapsulated
% postscript (.eps) format. pdflatex in pdf mode supports graphics
% in .pdf, .jpeg, .png and .mps (metapost) formats. Users should ensure
% that all non-photo figures use a vector format (.eps, .pdf, .mps) and
% not a bitmapped formats (.jpeg, .png). The IEEE frowns on bitmapped formats
% which can result in "jaggedy"/blurry rendering of lines and letters as
% well as large increases in file sizes.
%
% You can find documentation about the pdfTeX application at:
% http://www.tug.org/applications/pdftex





% *** MATH PACKAGES ***
%
%\usepackage{amsmath}
% A popular package from the American Mathematical Society that provides
% many useful and powerful commands for dealing with mathematics.
%
% Note that the amsmath package sets \interdisplaylinepenalty to 10000
% thus preventing page breaks from occurring within multiline equations. Use:
%\interdisplaylinepenalty=2500
% after loading amsmath to restore such page breaks as IEEEtran.cls normally
% does. amsmath.sty is already installed on most LaTeX systems. The latest
% version and documentation can be obtained at:
% http://www.ctan.org/pkg/amsmath





% *** SPECIALIZED LIST PACKAGES ***
%
%\usepackage{algorithmic}
% algorithmic.sty was written by Peter Williams and Rogerio Brito.
% This package provides an algorithmic environment fo describing algorithms.
% You can use the algorithmic environment in-text or within a figure
% environment to provide for a floating algorithm. Do NOT use the algorithm
% floating environment provided by algorithm.sty (by the same authors) or
% algorithm2e.sty (by Christophe Fiorio) as the IEEE does not use dedicated
% algorithm float types and packages that provide these will not provide
% correct IEEE style captions. The latest version and documentation of
% algorithmic.sty can be obtained at:
% http://www.ctan.org/pkg/algorithms
% Also of interest may be the (relatively newer and more customizable)
% algorithmicx.sty package by Szasz Janos:
% http://www.ctan.org/pkg/algorithmicx




% *** ALIGNMENT PACKAGES ***
%
%\usepackage{array}
% Frank Mittelbach's and David Carlisle's array.sty patches and improves
% the standard LaTeX2e array and tabular environments to provide better
% appearance and additional user controls. As the default LaTeX2e table
% generation code is lacking to the point of almost being broken with
% respect to the quality of the end results, all users are strongly
% advised to use an enhanced (at the very least that provided by array.sty)
% set of table tools. array.sty is already installed on most systems. The
% latest version and documentation can be obtained at:
% http://www.ctan.org/pkg/array


% IEEEtran contains the IEEEeqnarray family of commands that can be used to
% generate multiline equations as well as matrices, tables, etc., of high
% quality.




% *** SUBFIGURE PACKAGES ***
%\ifCLASSOPTIONcompsoc
%  \usepackage[caption=false,font=normalsize,labelfont=sf,textfont=sf]{subfig}
%\else
%  \usepackage[caption=false,font=footnotesize]{subfig}
%\fi
% subfig.sty, written by Steven Douglas Cochran, is the modern replacement
% for subfigure.sty, the latter of which is no longer maintained and is
% incompatible with some LaTeX packages including fixltx2e. However,
% subfig.sty requires and automatically loads Axel Sommerfeldt's caption.sty
% which will override IEEEtran.cls' handling of captions and this will result
% in non-IEEE style figure/table captions. To prevent this problem, be sure
% and invoke subfig.sty's "caption=false" package option (available since
% subfig.sty version 1.3, 2005/06/28) as this is will preserve IEEEtran.cls
% handling of captions.
% Note that the Computer Society format requires a larger sans serif font
% than the serif footnote size font used in traditional IEEE formatting
% and thus the need to invoke different subfig.sty package options depending
% on whether compsoc mode has been enabled.
%
% The latest version and documentation of subfig.sty can be obtained at:
% http://www.ctan.org/pkg/subfig




% *** FLOAT PACKAGES ***
%
%\usepackage{fixltx2e}
% fixltx2e, the successor to the earlier fix2col.sty, was written by
% Frank Mittelbach and David Carlisle. This package corrects a few problems
% in the LaTeX2e kernel, the most notable of which is that in current
% LaTeX2e releases, the ordering of single and double column floats is not
% guaranteed to be preserved. Thus, an unpatched LaTeX2e can allow a
% single column figure to be placed prior to an earlier double column
% figure.
% Be aware that LaTeX2e kernels dated 2015 and later have fixltx2e.sty's
% corrections already built into the system in which case a warning will
% be issued if an attempt is made to load fixltx2e.sty as it is no longer
% needed.
% The latest version and documentation can be found at:
% http://www.ctan.org/pkg/fixltx2e


%\usepackage{stfloats}
% stfloats.sty was written by Sigitas Tolusis. This package gives LaTeX2e
% the ability to do double column floats at the bottom of the page as well
% as the top. (e.g., "\begin{figure*}[!b]" is not normally possible in
% LaTeX2e). It also provides a command:
%\fnbelowfloat
% to enable the placement of footnotes below bottom floats (the standard
% LaTeX2e kernel puts them above bottom floats). This is an invasive package
% which rewrites many portions of the LaTeX2e float routines. It may not work
% with other packages that modify the LaTeX2e float routines. The latest
% version and documentation can be obtained at:
% http://www.ctan.org/pkg/stfloats
% Do not use the stfloats baselinefloat ability as the IEEE does not allow
% \baselineskip to stretch. Authors submitting work to the IEEE should note
% that the IEEE rarely uses double column equations and that authors should try
% to avoid such use. Do not be tempted to use the cuted.sty or midfloat.sty
% packages (also by Sigitas Tolusis) as the IEEE does not format its papers in
% such ways.
% Do not attempt to use stfloats with fixltx2e as they are incompatible.
% Instead, use Morten Hogholm'a dblfloatfix which combines the features
% of both fixltx2e and stfloats:
%
% \usepackage{dblfloatfix}
% The latest version can be found at:
% http://www.ctan.org/pkg/dblfloatfix




%\ifCLASSOPTIONcaptionsoff
%  \usepackage[nomarkers]{endfloat}
% \let\MYoriglatexcaption\caption
% \renewcommand{\caption}[2][\relax]{\MYoriglatexcaption[#2]{#2}}
%\fi
% endfloat.sty was written by James Darrell McCauley, Jeff Goldberg and 
% Axel Sommerfeldt. This package may be useful when used in conjunction with 
% IEEEtran.cls'  captionsoff option. Some IEEE journals/societies require that
% submissions have lists of figures/tables at the end of the paper and that
% figures/tables without any captions are placed on a page by themselves at
% the end of the document. If needed, the draftcls IEEEtran class option or
% \CLASSINPUTbaselinestretch interface can be used to increase the line
% spacing as well. Be sure and use the nomarkers option of endfloat to
% prevent endfloat from "marking" where the figures would have been placed
% in the text. The two hack lines of code above are a slight modification of
% that suggested by in the endfloat docs (section 8.4.1) to ensure that
% the full captions always appear in the list of figures/tables - even if
% the user used the short optional argument of \caption[]{}.
% IEEE papers do not typically make use of \caption[]'s optional argument,
% so this should not be an issue. A similar trick can be used to disable
% captions of packages such as subfig.sty that lack options to turn off
% the subcaptions:
% For subfig.sty:
% \let\MYorigsubfloat\subfloat
% \renewcommand{\subfloat}[2][\relax]{\MYorigsubfloat[]{#2}}
% However, the above trick will not work if both optional arguments of
% the \subfloat command are used. Furthermore, there needs to be a
% description of each subfigure *somewhere* and endfloat does not add
% subfigure captions to its list of figures. Thus, the best approach is to
% avoid the use of subfigure captions (many IEEE journals avoid them anyway)
% and instead reference/explain all the subfigures within the main caption.
% The latest version of endfloat.sty and its documentation can obtained at:
% http://www.ctan.org/pkg/endfloat
%
% The IEEEtran \ifCLASSOPTIONcaptionsoff conditional can also be used
% later in the document, say, to conditionally put the References on a 
% page by themselves.




% *** PDF, URL AND HYPERLINK PACKAGES ***
%
%\usepackage{url}
% url.sty was written by Donald Arseneau. It provides better support for
% handling and breaking URLs. url.sty is already installed on most LaTeX
% systems. The latest version and documentation can be obtained at:
% http://www.ctan.org/pkg/url
% Basically, \url{my_url_here}.




% *** Do not adjust lengths that control margins, column widths, etc. ***
% *** Do not use packages that alter fonts (such as pslatex).         ***
% There should be no need to do such things with IEEEtran.cls V1.6 and later.
% (Unless specifically asked to do so by the journal or conference you plan
% to submit to, of course. )


% correct bad hyphenation here
\hyphenation{op-tical net-works semi-conduc-tor}


\begin{document}
%
% paper title
% Titles are generally capitalized except for words such as a, an, and, as,
% at, but, by, for, in, nor, of, on, or, the, to and up, which are usually
% not capitalized unless they are the first or last word of the title.
% Linebreaks \\ can be used within to get better formatting as desired.
% Do not put math or special symbols in the title.
\title{Investigation of RISC-V ISA\\using Rocket-Chip}

%
%
% author names and IEEE memberships
% note positions of commas and nonbreaking spaces ( ~ ) LaTeX will not break
% a structure at a ~ so this keeps an author's name from being broken across
% two lines.
% use \thanks{} to gain access to the first footnote area
% a separate \thanks must be used for each paragraph as LaTeX2e's \thanks
% was not built to handle multiple paragraphs
%

\author{Ian Swepston, Ashton Johnson}% <-this % stops a space
\thanks{Department of Electrical Engineering and Computer Sciences, University of California, Berkeley, California, 94720}% <-this % stops a space
\thanks{Prof. Alexsander Milenkovic with the Department of Electrical and \\Computer Engineering, University of Alabama in Huntsville, Huntsville, AL 35899}% <-this % stops a space

% note the % following the last \IEEEmembership and also \thanks - 
% these prevent an unwanted space from occurring between the last author name
% and the end of the author line. i.e., if you had this:
% 
% \author{....lastname \thanks{...} \thanks{...} }
%                     ^------------^------------^----Do not want these spaces!
%
% a space would be appended to the last name and could cause every name on that
% line to be shifted left slightly. This is one of those "LaTeX things". For
% instance, "\textbf{A} \textbf{B}" will typeset as "A B" not "AB". To get
% "AB" then you have to do: "\textbf{A}\textbf{B}"
% \thanks is no different in this regard, so shield the last } of each \thanks
% that ends a line with a % and do not let a space in before the next \thanks.
% Spaces after \IEEEmembership other than the last one are OK (and needed) as
% you are supposed to have spaces between the names. For what it is worth,
% this is a minor point as most people would not even notice if the said evil
% space somehow managed to creep in.



% The paper headers
\markboth{CPE 631, SPRING 2017}%
{Shell \MakeLowercase{\textit{et al.}}: Bare Demo of IEEEtran.cls for IEEE Journals}
% The only time the second header will appear is for the odd numbered pages
% after the title page when using the twoside option.
% 
% *** Note that you probably will NOT want to include the author's ***
% *** name in the headers of peer review papers.                   ***
% You can use \ifCLASSOPTIONpeerreview for conditional compilation here if
% you desire.

% If you want to put a publisher's ID mark on the page you can do it like
% this:
%\IEEEpubid{0000--0000/00\$00.00~\copyright~2015 IEEE}
% Remember, if you use this you must call \IEEEpubidadjcol in the second
% column for its text to clear the IEEEpubid mark.

% use for special paper notices
%\IEEEspecialpapernotice{(Invited Paper)}

% make the title area
\maketitle

% As a general rule, do not put math, special symbols or citations
% in the abstract or keywords.
\begin{abstract}
RISC-V is a new and open RISC instruction set architecture (ISA). There are a series of repositories provided by The University of California, Berkeley to support the ISA help to design hardware. In this paper we investigate these tools and use them to learn more about RISC-V and RISC-V processors. 
\end{abstract}

% Note that keywords are not normally used for peerreview papers.
\begin{IEEEkeywords}
RISC-V, ISA, Zedboard, Zynq, Rocket-Chip
\end{IEEEkeywords}

% For peer review papers, you can put extra information on the cover
% page as needed:
% \ifCLASSOPTIONpeerreview
% \begin{center} \bfseries EDICS Category: 3-BBND \end{center}
% \fi
%
% For peerreview papers, this IEEEtran command inserts a page break and
% creates the second title. It will be ignored for other modes.
%\IEEEpeerreviewmaketitle



\section{Introduction}
% The very first letter is a 2 line initial drop letter followed
% by the rest of the first word in caps.
% 
% form to use if the first word consists of a single letter:
% \IEEEPARstart{A}{demo} file is ....
% 
% form to use if you need the single drop letter followed by
% normal text (unknown if ever used by the IEEE):
% \IEEEPARstart{A}{}demo file is ....
% 
% Some journals put the first two words in caps:
% \IEEEPARstart{T}{his demo} file is ....
% 
% Here we have the typical use of a "T" for an initial drop letter
% and "HIS" in caps to complete the first word.
\IEEEPARstart{T}{he} RISC-V open instruction set architecture commissioned in 2010 has been making waves in the in world of open hardware within both the academic community and industry. As students pursuant of masters degrees in computer engineering, we investigate the RISC-V processor through venues that we are already familiar with: Field Programmable Gate Arrays (FGPAs).
\newline The Rocket-Chip generator allows for developing a system-on-chip (SOC) design rapidly.\cite{Asanović:EECS-2016-17}. We utilize the fpga-zynq \cite{fpga-zynq} repository to target a Zynq-based Zedboard FPGA development platform to instantiate and test out a RISC-V implementation.
% You must have at least 2 lines in the paragraph with the drop letter
% (should never be an issue)

%\hfill iXs & acj
%\hfill August 26, 2015

\section{Project Goals}
The goals of our project are to investigate the source material provided by UC Berkeley's Rocket Chip Generator. Specifically, we will attempt to replicate the successes of those listed in the fpga-zynq \cite{fpga-zynq} repository. Goals are listed in order or complexity, but also different layers of the supplied design. 
% needed in second column of first page if using \IEEEpubid
%\IEEEpubidadjcol
\begin{itemize}
\item{Load Prebuilt Image}%TODO determine to to remove colons generated
\item{Modify FPGA Project}
\item{Modify RISC-V ISA}
\item{Document Results}
\end{itemize}

\section{RISC-V Overview}
RISC-V is an open risc instruction set architecture (ISA) commissioned in 2010 by a research team at UC Berkeley \cite{Waterman:EECS-2016-118}. The stated goal of the team was to create a free and open ISA for commercial, educational, and open source settings. RISC-V is governed by the RISC-V foundation and is under the BSD (Berkeley Software Distribution) Licence, which is similar to the LGPL. The foundation is intended to maintain the ISA in the long term and guide it toward being an industry standard.\newline
RISC-V is not the first attempt at an open ISA, but RISC-V appears to be positioned to succeed where others failed. The RISC-V foundation is important. They can act in the same way a company, like ARM Holdings, and will continue to improve the ISA, but will not monetizing it. Free hardware designs are also helpful. Other open ISAs have failed because they were impractical to implement. Potential adopters needed to either design their own cores or try to implement untested designs. Finally, educators are eager to adopt this ISA. The possibility of free-to-use ISA and hardware designs would allow us to use real assembly and real hardware to teach architecture. This would create demand and would flood the design world with scientists and engineers that know the ISA and will choose to use it due to familiarity.

% Generates some bullet points
\subsection{Design Benefits}
\begin{itemize}
\item BSD License:\newline
  Allows proprietary derivative work
\item Scaleability:\newline
  The ISA is largely microarchitecture-agnostic and flexible enough for most application workloads. It can be scaled to cover everything from microcontrollers to warehouse-scale servers.
\item Code Compression:\newline
  The ISA supports compression of its binary programs down to 16 bits per command without losing core functionality. This allows for significant reductions in code size for small scale embedded devices.
\item 128 Bit:\newline
  Comes with fully-functional 128-bit design. This does not appear to be commercially supported and may be of use soon in large-scale cloud computing applications.
\item Software Support:\newline
  There is a free software suite for application development. This will be discussed later.
\end{itemize}

\section{Pre-Built Image}
It is possible to load a pre-built RISC-V core design onto a Xilinx Zedboard. The image is provided in the fpga-zynq repository. This image contains a basic implementation that allows the toolchain to be tested and for the execution of user programs, such as benchmarks. This configuration was investigated and a simple program was compiled using the toolchain and was executed on the core. We were able to compile and run the basic matrix multiplication program we developed in the early part of the CPE 631 class. It should be noted that this program utilized double precision floating points numbers.

\subsection{Image Makeup}
The pre-built configuration is a simple design that can be instantiated on the Zedboard. It is a single core design for the purpose of rapidly testing and running RISC-V in hardware. The basic design is an RV64G core. This means that it is a RISC-V (RV) 64 bit core servicing the General Purpose (G) extension set. The general purpose set can be thought of as a ``normal'' processor. We will cover the details in the Rocket Chip Generator section later. \newline
Surrounding the core in this design is a split L1 cache for instructions and data. These caches are designed to be simple and save FPGA space. Both are 4-way set associative with 64 sets per cache and 8 TLB entries. They are non-error correcting, use the random policy for replacement, and do not use scratchpad memory to store calculations. \newline
This design does not contain an L2 cache due to size and complexity, so after L1, the processor must go directly to main memory. For main memory, the design splits the 512MB of on-board Zedboard DDR memory in half, utilizing 256MB for its needs and the other half allocated to the ARM cores in the Zynq.

\subsection{Toolchain}
To complete this goal, cross-compilation was necessary. For this, Berkeley has developed and released the sources for RISC-V variants of the LLVM and GNU toolchains. These are included in the RISC-V repository under fpga-zynq/rocket-chip/riscv-tools. They are available as a standalone or as a link within the fpga-zynq project for Zynq boards. The toolchains can be installed several ways, but for performance and simple installation, we recommend using the provided build scripts in the riscv-tools directory. Instructions for installing the toolchain can be found in risc-gnu-toolchain repository \cite{toolchain}. Once built on the native system, the tools are named like: riscv64-unknown-elf-g++ and can be installed anywhere on that system. These toolchains appear to be complete and operate identically to their mainstream versions. As an example, compilation can be completed using the normal formats: [path to tools]/riscv64-unknown-elf-g++ main.cpp -o executable. \newline \newline While the toolchains are being built, some additional useful tools are constructed:\newline

\subsubsection{Spike and pk}
The executable for spike is built here. Spike is the RISC-V open source cross-platform simulator designed by Berkeley. It allows for execution of RISC-V binaries on an arbitrary platform through emulation. This is paired with a program called pk (proxy kernel). pk is designed to format and provide RISC-V instructions to either a kernel-less core or the spike program for execution and to receive and handle output from the core. \newline

\subsubsection{Benchmarks}
Embedded in this repository are a series of RISC-V benchmarks compiled and ready to be executed on a system with a RISC-V core. 

\section{FPGA Project}
Using the supplied makefiles, a Xilinx Vivado project was generated according to the the default Rocket Chip configuration. From this,  we investigated the design that was targeting the Zedboard.
On the Zedboard, the Rocket core interfaces with the ARM processing system via two AXI interfaces. One AXI interace is provides the ARM the ability to load user executibles into the rocket core. The other AXI interface provides the rocket core access to the DRAM that is on the zedboard. The memory side interface utilizes one of th AXI-HP ports on the Zynq Processing Module that is instantiated in a Block Diagram.

\begin{figure} %reference https://en.wikibooks.org/wiki/LaTeX/Floats,_Figures_and_Captions for [!h] options on placing the figures
\centering
\includegraphics[width=9cm ]{./graphics/zynq_w_gpio.png}
\centering
\caption{Design with GPIO peripheral.}
\label{fig:gpio}
\end{figure}
In Fig~\ref{fig:gpio}, we attached a Xilinx supplied AXI GPIO IP core. At the time, this method seemed to be the easiest for making a peripheral available to the RISC-V processor. The GPIO could then be accessed as a memory-mapped IO device using an address generated by the Xilinx block diagram. While we were successful in generating a new bitstream for the zedboard, and loading the design onto the board, we were not successful in actually manipulating the GPIO device. When running an executable using the Proxy Kernel\cite{pk}, an error is reported of a 'user store segfault' at the address of the GPIO. This result was also confirmed using the Spike simulator. Upon researching further, there are similar efforts to attach peripherals to the processor as you might expect. We discovered that the lowRISC project provides an exampled on how peripherals should be added\cite{lowrisc}. Using the diagrams they provided, we now understand that peripherals should be connected via non-cachable channels. This would require a unique AXI based bus from the processor that doesn't route through the memory interconnects from the L1 cache to the main memory. We were not able to pursue this further given the time 

\section{Rocket Chip Generator}
In order to rapidly produce prototypes of a design and to make RISC-V more accessible, the Berkeley team has created a system called the Rocket Chip generator. By using this tool, a hardware architect can have a RISC-V core up and running within minutes of downloading the appropriate repository and can begin to make changes as they are needed.

\subsection{Chisel}
The Rocket Chip generator relies on a Berkeley developed language extension of the Scala programming language, called Chisel\cite{bachrach2012chisel}. Using Chisel, functional blocks of hardware are described for all the components required in the rocket core. Chisel is a new breed of hardware description language. Compared to traditional HDLs, such as Verilog or VDHL, Chisel can be considered a higher level language. It is syntactically similar to Java and employs many Object Oriented principles such as classes, inheritance and polymorphism. Within Chisel, hardware configurations are developed by instantiating base modules, which are treated like objects, and then including options by overriding the default parameters of each module. The hardware is described and connected hierarchically and upon evaluation the Chisel code generates Verilog which may be used in synthesis.

\begin{figure} %reference https://en.wikibooks.org/wiki/LaTeX/Floats,_Figures_and_Captions for [!h] options on placing the figures
\centering
\includegraphics[width=9cm ]{./graphics/ExampleRocketChip.png}
\centering
\caption{Architecture that may be generated by the Rocket Chip Generator\cite{Asanović:EECS-2016-17}}
\label{fig:rocketchip}
\end{figure}

\subsection{Generation Options}
The Rocket Chip Generator is intended to be a do-it-all hardware design tool for hardware architects. As such, it is very flexible. Fig~\ref{fig:rocketchip} is an example of an SoC created by the Rocket Chip generator and it demonstrates the power of the tool's sub-generators working together. For example, this instance could be generated and contains two separate, self-contained procesors connected by an interconnection network and L2 cache plus an AXI4 Crossbar with peripherals like a memory controller attached. The processor in Tile1 contains a small coprocessor and Berkeleys custom-made out of order RISC-V core (BOOM) and Tile2 has a general Rocket core for basic operations and a larger coprocessor.

\subsection{Extensions}
Extensions are a mechanism used by the chip generator to allow easy manipulation of the core and control the supported section of the instruction set. Rocket cores can be as simple or as complex as needed to complete tasks. Much of this complexity is due to the inclusion and exclusion of extensions. For convenience, many commonly used extensions are available to designers pre-written within the generator. The ``G'' extension is a combination of the extension for Integer (I), Multiply/Divide (M), Atomic (A), Floating Point (F), and Double Precision (D). This could be known as RV64IMAFD, but because these are so common, they are combined as G and referenced as RV64G \cite{Waterman:EECS-2016-118}. 

\section{Rocket Chip Modification}
The first step in understanding the generator is attempting to modify the design. This was our goal. Using chisel we successfully modified the number of cores instantiated in the design. This can easily be performed by modifying the following text in the file [path-to-repo]/rocket-chip/src/main/scala/rocketchip/Configs.scala to include the bolded text:
\newline
\newline
\tt{class DefaultFPGAConfig extends \textbf{++ new WithNCores(2)} Config(new FPGAConfig ++ new BaseConfig)}\rm{}
\newline
\newline
While generating a dual core configuration was simple, it was not we could not implement this on the Zedboard because the resource utilization was too great. This was due to two reasons:
\subsection{LUT Utilization}
The processing coreplexes take up around 60 percent of the device LUT utilization. Therefore two core utilize 120 percent. Using a larger device could mitigate this problem.
\subsection{Block RAM Utilization}
When instantiating two or more cores, an L2 cache is automatically inferred. This is quite convenient; however, the default L2 instantiation allocates 2048KB of memory. This required around 1,200 percent of Block RAM utilization! To resolve this, we modified the previous snippet by adding the follow bold text:
\newline
\newline
\tt{class DefaultFPGAConfig extends ++ new WithNCores(2) \textbf{+ new WithL2Capacity(256)} Config(new FPGAConfig ++ new BaseConfig)}\rm{}
\newline
\newline
To faciliate testing out the different configurations, we defined several classes beyond the \tt{DefaultFPGAConfig}\rm{} that can be instantiated. Their names are self-explanatory:
\newline
\newline
\tt{
class DefaultFPGARoccConfig \newline
class DefaultFPGA32KBL2Config \newline
class DefaultFPGA256KBL2Config \newline
class DefaultFPGADualCoreConfig}\rm{}

\section{RoCC Accelerator}
The Rocket-Chip generator provides a framework for developeing custom co-processors that are housed in the processing coreplex, allowing for low latency access to the co-processor. These are the Rocket Custom Coprocessor (RoCC). These coprocessors utilize reserved opcodes in the instruction set. To get started with the accelerators, three example coprocessors are already defined in the repository. These are defined down in .../rocket/rocc.Scala. We were sucessful in instantiating these cores, and provided \tt{class DefaultFPGARoccConfig}\rm{} to make these easier to implmement. Resource utilization of the custom coprocessors is negligible.

We ran into difficulties when developing software for targeting the custom coprocessors. Documentation for how to target these is absent, so we were left to seaching user forums on the internet to see guidance. After much research, we were still not sucessful in using the custom opcodes. We discovered in previous iterations of the gnu-toolchain, the custome(0/1/2/3) opcodes were supported. Since then, the team has removed default support for the opcodes, and now require the use of pseudo opcodes\cite{psuedocode}. 

\section{Notable Failures}
Throughout this process, we experienced success; however, we experienced even more failures. We note these so as not to disguise, but to bring forward issues with the system, so that follow-on efforts might not be able to resolve them, or at least be aware of them.

\subsubsection{Custom Opcodes}
As mentioned previously, we were unsuccessful in utilizing custom opcodes, even those designed as pre-planned custom opcodes. While we were finally able to compile programs using the pseudo opcodes technique, the proxy kernel would not allow the execution of those instructions. They were designated as illegal opcodes. We suspect the proxy kernel would need to be recompiled to allow these instructions. 

\subsubsection{Deploying Linux}
Through multiple paths we were not able to rebuild GNU/Linux and execute it on the processor. This included both on the development board and the Spike simulator. To give background to this, The Berkeley Bootloader (BBL), which is found in the riscv-pk repository\cite{pk}, is an executable that is used to load in an ELF file into the processor. Most notably, this is used to boot GNU/Linux. When compiling the BBL, the Linux ELF file must be specified, otherwise and dummy ELF file is included. The output of the BBL build process is a single executable that contains the kernel.

We were successful in building the Linux kernel using the riscv-linux repository\cite{linux}. Additionally, we were successful in building a GNU/Linux distribution using the Yocto Pocky and the recipes provided in riscv-poky\cite{poky}. For Poky, we validated the build using the Spike simulator and BBL that is compiled at alongside the kernel. Unfortunately, moving that working configuration of the BLL and the kernel image and rootfs to the SD card did not yield success. Reflecting more on this configuration, it appears that version of the BBL does not require the kernel to be compiled into it. It seems to be an argument to it.

\subsubsection{AXI Peripherals}
We were not able to access a memory mapped I/O device. In this case a GPIO peripheral. Our test code was written to directly read and write to a pointer-address location (0x4000 0000). Yet again, the proxy kernel reported a segfault when reaching this step of the program. This restriction could be removed with recompilation of the proxy kernel, but this is only speculation.

\section{Document Results}
For documentation we are providing an Oracle VMWare disk image to reach each of the three goals listed above using the virtual machine provided. The disk image is an Ubuntu 16.04 distribution with all the required repositories and examples necessary to replicate our goals. We felt this was the best approach for allowing others to verify and replicate our results. Additionally, instructions for the virtual machine getting a new virtual machine up and running using the virtual disk provided. While this will not provided the fastest avenue for building the different items required, it provided a path with all dependencies removed. 


% An example of a floating figure using the graphicx package.
% Note that \label must occur AFTER (or within) \caption.
% For figures, \caption should occur after the \includegraphics.
% Note that IEEEtran v1.7 and later has special internal code that
% is designed to preserve the operation of \label within \caption
% even when the captionsoff option is in effect. However, because
% of issues like this, it may be the safest practice to put all your
% \label just after \caption rather than within \caption{}.
%
% Reminder: the "draftcls" or "draftclsnofoot", not "draft", class
% option should be used if it is desired that the figures are to be
% displayed while in draft mode.
%
%\begin{figure}[!t]
%\centering
%\includegraphics[width=2.5in]{myfigure}
% where an .eps filename suffix will be assumed under latex, 
% and a .pdf suffix will be assumed for pdflatex; or what has been declared
% via \DeclareGraphicsExtensions.
%\caption{Simulation results for the network.}
%\label{fig_sim}
%\end{figure}

% Note that the IEEE typically puts floats only at the top, even when this
% results in a large percentage of a column being occupied by floats.


% An example of a double column floating figure using two subfigures.
% (The subfig.sty package must be loaded for this to work.)
% The subfigure \label commands are set within each subfloat command,
% and the \label for the overall figure must come after \caption.
% \hfil is used as a separator to get equal spacing.
% Watch out that the combined width of all the subfigures on a 
% line do not exceed the text width or a line break will occur.
%
%\begin{figure*}[!t]
%\centering
%\subfloat[Case I]{\includegraphics[width=2.5in]{box}%
%\label{fig_first_case}}
%\hfil
%\subfloat[Case II]{\includegraphics[width=2.5in]{box}%
%\label{fig_second_case}}
%\caption{Simulation results for the network.}
%\label{fig_sim}
%\end{figure*}
%
% Note that often IEEE papers with subfigures do not employ subfigure
% captions (using the optional argument to \subfloat[]), but instead will
% reference/describe all of them (a), (b), etc., within the main caption.
% Be aware that for subfig.sty to generate the (a), (b), etc., subfigure
% labels, the optional argument to \subfloat must be present. If a
% subcaption is not desired, just leave its contents blank,
% e.g., \subfloat[].


% An example of a floating table. Note that, for IEEE style tables, the
% \caption command should come BEFORE the table and, given that table
% captions serve much like titles, are usually capitalized except for words
% such as a, an, and, as, at, but, by, for, in, nor, of, on, or, the, to
% and up, which are usually not capitalized unless they are the first or
% last word of the caption. Table text will default to \footnotesize as
% the IEEE normally uses this smaller font for tables.
% The \label must come after \caption as always.
%
%\begin{table}[!t]
%% increase table row spacing, adjust to taste
%\renewcommand{\arraystretch}{1.3}
% if using array.sty, it might be a good idea to tweak the value of
% \extrarowheight as needed to properly center the text within the cells
%\caption{An Example of a Table}
%\label{table_example}
%\centering
%% Some packages, such as MDW tools, offer better commands for making tables
%% than the plain LaTeX2e tabular which is used here.
%\begin{tabular}{|c||c|}
%\hline
%One & Two\\
%\hline
%Three & Four\\
%\hline
%\end{tabular}
%\end{table}


% Note that the IEEE does not put floats in the very first column
% - or typically anywhere on the first page for that matter. Also,
% in-text middle ("here") positioning is typically not used, but it
% is allowed and encouraged for Computer Society conferences (but
% not Computer Society journals). Most IEEE journals/conferences use
% top floats exclusively. 
% Note that, LaTeX2e, unlike IEEE journals/conferences, places
% footnotes above bottom floats. This can be corrected via the
% \fnbelowfloat command of the stfloats package.

\section{Conclusion}
RISC-V is an open ISA designed to break open the world of computer architecture. It is making the push to be the first successful open ISA. The ISA is surrounded by many advantages including a software toolchain and a core design tool to generate inexpensive and reliable hardware. This paper documents our investigation into this ISA and its supporting tools. During the course of the paper we have discussed how we tested these tools utilizing the existing documentation from UC Berkeley for the purpose of improving upon it and understanding RISC-V.

We were able to load and test the pre-built designs in their repositories as well as complete their examples. We were also able to learn about the Rocket Chip generator tool and Chisel, its supporting language. Using these we were able to build a basic rocket core from source. Also using the generator, we made modifications to the core design and configuration and synthesized these cores. We also investigated adding a coprocessor (called a RoCC) and how to add it to a core design. We were able to synthesize this as well. 

We had a few failures along the way. Many of these were due to the state of the repositories being in flux or due to venturing into underdocumented or undocumented pieces of the Rocket Chip generator and RISC-V. We believe that these failures can bring to light information that needs future investigation and documentation. 

It is these authors opinion that RISC-V is poised for success. It is at a critical time in which it is being presented to the world, picked apart, and tested. As such we feel that researching and understanding the ISA and the tools that have been created to support it will help expand it use and usability.

% if have a single appendix:
%\appendix[Proof of the Zonklar Equations]
% or
%\appendix  % for no appendix heading
% do not use \section anymore after \appendix, only \section*
% is possibly needed

% use appendices with more than one appendix
% then use \section to start each appendix
% you must declare a \section before using any
% \subsection or using \label (\appendices by itself
% starts a section numbered zero.)
%


% use section* for acknowledgment
\section*{Acknowledgment}

The authors would like to thank the UC Berkeley Architecture Research team for the use of their materials and research. Thanks are also due to the RISC-V foundation and the open source community for their documentation of the RISC-V ISA.


% Can use something like this to put references on a page
% by themselves when using endfloat and the captionsoff option.
\ifCLASSOPTIONcaptionsoff
  \newpage
\fi



% trigger a \newpage just before the given reference
% number - used to balance the columns on the last page
% adjust value as needed - may need to be readjusted if
% the document is modified later
%\IEEEtriggeratref{8}
% The "triggered" command can be changed if desired:
%\IEEEtriggercmd{\enlargethispage{-5in}}

% references section

% can use a bibliography generated by BibTeX as a .bbl file
% BibTeX documentation can be easily obtained at:
% http://mirror.ctan.org/biblio/bibtex/contrib/doc/
% The IEEEtran BibTeX style support page is at:
% http://www.michaelshell.org/tex/ieeetran/bibtex/
\bibliographystyle{IEEEtran}
% argument is your BibTeX string definitions and bibliography database(s)
\bibliography{cpe631}
%
% <OR> manually copy in the resultant .bbl file
% set second argument of \begin to the number of references
% (used to reserve space for the reference number labels box)
%\begin{thebibliography}{1}
 
%\bibitem{Asanović:EECS-2016-17}
 
%\end{thebibliography}

% biography section
% 
% If you have an EPS/PDF photo (graphicx package needed) extra braces are
% needed around the contents of the optional argument to biography to prevent
% the LaTeX parser from getting confused when it sees the complicated
% \includegraphics command within an optional argument. (You could create
% your own custom macro containing the \includegraphics command to make things
% simpler here.)
%\begin{IEEEbiography}[{\includegraphics[width=1in,height=1.25in,clip,keepaspectratio]{mshell}}]{Michael Shell}
% or if you just want to reserve a space for a photo:

\begin{IEEEbiographynophoto}{Ian Swepston}
 Received his B.S. in Computer Engineering from Clemson University in 2015. He is currently pursuing a Master's Degree in Computer Engineering at the University of Alabama in Huntsville. He currently works as a Computer Engineer at Dynetics, Inc. in Huntsville.
\end{IEEEbiographynophoto}

% if you will not have a photo at all:
\begin{IEEEbiographynophoto}{Ashton Johnson}
 received his B.E. degree in wireless engineering from Auburn University in 2011. He is currently seeking his Master's degree in computer engineering at the University of Alabama in Huntsville. He current works as an electrical engineer at Dynetics, Inc. in Huntsville. 
\end{IEEEbiographynophoto}

% insert where needed to balance the two columns on the last page with
% biographies
%\newpage

% You can push biographies down or up by placing
% a \vfill before or after them. The appropriate
% use of \vfill depends on what kind of text is
% on the last page and whether or not the columns
% are being equalized.

%\vfill

% Can be used to pull up biographies so that the bottom of the last one
% is flush with the other column.
%\enlargethispage{-5in}



% that's all folks
\end{document}



%%  LocalWords:  DefaultFPGAConfig WithNCores Config FPGAConfig
%%  LocalWords:  BaseConfig

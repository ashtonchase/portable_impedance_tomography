
%% cpe621.tex
%% 2017/07/20
%% by Ashton Johnson 
%% see http://github.com/ashtonchase/portable_impedance_tomograhy
%% for current contact information.
%%
%% This is a skeleton file demonstrating the use of IEEEtran.cls
%% (requires IEEEtran.cls version 1.8b or later) with an IEEE
%% journal paper.
%%
%% Support sites:
%% http://www.michaelshell.org/tex/ieeetran/
%% http://www.ctan.org/pkg/ieeetran
%% and
%% http://www.ieee.org/

%%*************************************************************************
%% Legal Notice:
%% This code is offered as-is without any warranty either expressed or
%% implied; without even the implied warranty of MERCHANTABILITY or
%% FITNESS FOR A PARTICULAR PURPOSE! 
%% User assumes all risk.
%% In no event shall the IEEE or any contributor to this code be liable for
%% any damages or losses, including, but not limited to, incidental,
%% consequential, or any other damages, resulting from the use or misuse
%% of any information contained here.
%%
%% All comments are the opinions of their respective authors and are not
%% necessarily endorsed by the IEEE.
%%
%% This work is distributed under the LaTeX Project Public License (LPPL)
%% ( http://www.latex-project.org/ ) version 1.3, and may be freely used,
%% distributed and modified. A copy of the LPPL, version 1.3, is included
%% in the base LaTeX documentation of all distributions of LaTeX released
%% 2003/12/01 or later.
%% Retain all contribution notices and credits.
%% ** Modified files should be clearly indicated as such, including  **
%% ** renaming them and changing author support contact information. **
%%*************************************************************************


% *** Authors should verify (and, if needed, correct) their LaTeX system  ***
% *** with the testflow diagnostic prior to trusting their LaTeX platform ***
% *** with production work. The IEEE's font choices and paper sizes can   ***
% *** trigger bugs that do not appear when using other class files.       ***
% The testflow support page is at:
% http://www.michaelshell.org/tex/testflow/



%\documentclass[journal]{IEEEtran}
%
% If IEEEtran.cls has not been installed into the LaTeX system files,
% manually specify the path to it like:
 \documentclass[]{IEEEtran}

% Some very useful LaTeX packages include:
% (uncomment the ones you want to load)

% *** MISC UTILITY PACKAGES ***
%
%\usepackage{ifpdf}
% Heiko Oberdiek's ifpdf.sty is very useful if you need conditional
% compilation based on whether the output is pdf or dvi.
% usage:
% \ifpdf
%   pdf code
% \else
%    dvi code
% \fi
% The latest version of ifpdf.sty can be obtained from:
% http://www.ctan.org/pkg/ifpdf
% Also, note that IEEEtran.cls V1.7 and later provides a builtin
% \ifCLASSINFOpdf conditional that works the same way.
% When switching from latex to pdflatex and vice-versa, the compiler may
% have to be run twice to clear warning/error messages.

% *** CITATION PACKAGES ***
%
\usepackage{cite}
% cite.sty was written by Donald Arseneau
% V1.6 and later of IEEEtran pre-defines the format of the cite.sty package
% \cite{} output to follow that of the IEEE. Loading the cite package will
% result in citation numbers being automatically sorted and properly
% "compressed/ranged". e.g., [1], [9], [2], [7], [5], [6] without using
% cite.sty will become [1], [2], [5]--[7], [9] using cite.sty. cite.sty's
% \cite will automatically add leading space, if needed. Use cite.sty's
% noadjust option (cite.sty V3.8 and later) if you want to turn this off
% such as if a citation ever needs to be enclosed in parenthesis.
% cite.sty is already installed on most LaTeX systems. Be sure and use
% version 5.0 (2009-03-20) and later if using hyperref.sty.
% The latest version can be obtained at:
% http://www.ctan.org/pkg/cite
% The documentation is contained in the cite.sty file itself.

% *** GRAPHICS RELATED PACKAGES ***
%
\ifCLASSINFOpdf
   \usepackage{graphicx}
  % declare the path(s) where your graphic files are
%   \graphicspath{{../pdf/}{../jpeg/}}
  % and their extensions so you won't have to specify these with
  % every instance of \includegraphics
   \DeclareGraphicsExtensions{.pdf,.jpeg,.png}
\else
  % or other class option (dvipsone, dvipdf, if not using dvips). graphicx
  % will default to the driver specified in the system graphics.cfg if no
  % driver is specified.
   \usepackage[pdftex]{graphicx}
  % declare the path(s) where your graphic files are
 %  \graphicspath{{../pdf/}}
  % and their extensions so you won't have to specify these with
  % every instance of \includegraphics
%   \DeclareGraphicsExtensions{.eps,.pdf}
\fi
% graphicx was written by David Carlisle and Sebastian Rahtz. It is
% required if you want graphics, photos, etc. graphicx.sty is already
% installed on most LaTeX systems. The latest version and documentation
% can be obtained at: 
% http://www.ctan.org/pkg/graphicx
% Another good source of documentation is "Using Imported Graphics in
% LaTeX2e" by Keith Reckdahl which can be found at:
% http://www.ctan.org/pkg/epslatex
%
% latex, and pdflatex in dvi mode, support graphics in encapsulated
% postscript (.eps) format. pdflatex in pdf mode supports graphics
% in .pdf, .jpeg, .png and .mps (metapost) formats. Users should ensure
% that all non-photo figures use a vector format (.eps, .pdf, .mps) and
% not a bitmapped formats (.jpeg, .png). The IEEE frowns on bitmapped formats
% which can result in "jaggedy"/blurry rendering of lines and letters as
% well as large increases in file sizes.
%
% You can find documentation about the pdfTeX application at:
% http://www.tug.org/applications/pdftex





% *** MATH PACKAGES ***
%
%\usepackage{amsmath}
% A popular package from the American Mathematical Society that provides
% many useful and powerful commands for dealing with mathematics.
%
% Note that the amsmath package sets \interdisplaylinepenalty to 10000
% thus preventing page breaks from occurring within multiline equations. Use:
%\interdisplaylinepenalty=2500
% after loading amsmath to restore such page breaks as IEEEtran.cls normally
% does. amsmath.sty is already installed on most LaTeX systems. The latest
% version and documentation can be obtained at:
% http://www.ctan.org/pkg/amsmath





% *** SPECIALIZED LIST PACKAGES ***
%
%\usepackage{algorithmic}
% algorithmic.sty was written by Peter Williams and Rogerio Brito.
% This package provides an algorithmic environment fo describing algorithms.
% You can use the algorithmic environment in-text or within a figure
% environment to provide for a floating algorithm. Do NOT use the algorithm
% floating environment provided by algorithm.sty (by the same authors) or
% algorithm2e.sty (by Christophe Fiorio) as the IEEE does not use dedicated
% algorithm float types and packages that provide these will not provide
% correct IEEE style captions. The latest version and documentation of
% algorithmic.sty can be obtained at:
% http://www.ctan.org/pkg/algorithms
% Also of interest may be the (relatively newer and more customizable)
% algorithmicx.sty package by Szasz Janos:
% http://www.ctan.org/pkg/algorithmicx




% *** ALIGNMENT PACKAGES ***
%
%\usepackage{array}
% Frank Mittelbach's and David Carlisle's array.sty patches and improves
% the standard LaTeX2e array and tabular environments to provide better
% appearance and additional user controls. As the default LaTeX2e table
% generation code is lacking to the point of almost being broken with
% respect to the quality of the end results, all users are strongly
% advised to use an enhanced (at the very least that provided by array.sty)
% set of table tools. array.sty is already installed on most systems. The
% latest version and documentation can be obtained at:
% http://www.ctan.org/pkg/array


% IEEEtran contains the IEEEeqnarray family of commands that can be used to
% generate multiline equations as well as matrices, tables, etc., of high
% quality.




% *** SUBFIGURE PACKAGES ***
%\ifCLASSOPTIONcompsoc
%  \usepackage[caption=false,font=normalsize,labelfont=sf,textfont=sf]{subfig}
%\else
%  \usepackage[caption=false,font=footnotesize]{subfig}
%\fi
% subfig.sty, written by Steven Douglas Cochran, is the modern replacement
% for subfigure.sty, the latter of which is no longer maintained and is
% incompatible with some LaTeX packages including fixltx2e. However,
% subfig.sty requires and automatically loads Axel Sommerfeldt's caption.sty
% which will override IEEEtran.cls' handling of captions and this will result
% in non-IEEE style figure/table captions. To prevent this problem, be sure
% and invoke subfig.sty's "caption=false" package option (available since
% subfig.sty version 1.3, 2005/06/28) as this is will preserve IEEEtran.cls
% handling of captions.
% Note that the Computer Society format requires a larger sans serif font
% than the serif footnote size font used in traditional IEEE formatting
% and thus the need to invoke different subfig.sty package options depending
% on whether compsoc mode has been enabled.
%
% The latest version and documentation of subfig.sty can be obtained at:
% http://www.ctan.org/pkg/subfig




% *** FLOAT PACKAGES ***
%
%\usepackage{fixltx2e}
% fixltx2e, the successor to the earlier fix2col.sty, was written by
% Frank Mittelbach and David Carlisle. This package corrects a few problems
% in the LaTeX2e kernel, the most notable of which is that in current
% LaTeX2e releases, the ordering of single and double column floats is not
% guaranteed to be preserved. Thus, an unpatched LaTeX2e can allow a
% single column figure to be placed prior to an earlier double column
% figure.
% Be aware that LaTeX2e kernels dated 2015 and later have fixltx2e.sty's
% corrections already built into the system in which case a warning will
% be issued if an attempt is made to load fixltx2e.sty as it is no longer
% needed.
% The latest version and documentation can be found at:
% http://www.ctan.org/pkg/fixltx2e


%\usepackage{stfloats}
% stfloats.sty was written by Sigitas Tolusis. This package gives LaTeX2e
% the ability to do double column floats at the bottom of the page as well
% as the top. (e.g., "\begin{figure*}[!b]" is not normally possible in
% LaTeX2e). It also provides a command:
%\fnbelowfloat
% to enable the placement of footnotes below bottom floats (the standard
% LaTeX2e kernel puts them above bottom floats). This is an invasive package
% which rewrites many portions of the LaTeX2e float routines. It may not work
% with other packages that modify the LaTeX2e float routines. The latest
% version and documentation can be obtained at:
% http://www.ctan.org/pkg/stfloats
% Do not use the stfloats baselinefloat ability as the IEEE does not allow
% \baselineskip to stretch. Authors submitting work to the IEEE should note
% that the IEEE rarely uses double column equations and that authors should try
% to avoid such use. Do not be tempted to use the cuted.sty or midfloat.sty
% packages (also by Sigitas Tolusis) as the IEEE does not format its papers in
% such ways.
% Do not attempt to use stfloats with fixltx2e as they are incompatible.
% Instead, use Morten Hogholm'a dblfloatfix which combines the features
% of both fixltx2e and stfloats:
%
% \usepackage{dblfloatfix}
% The latest version can be found at:
% http://www.ctan.org/pkg/dblfloatfix




%\ifCLASSOPTIONcaptionsoff
%  \usepackage[nomarkers]{endfloat}
% \let\MYoriglatexcaption\caption
% \renewcommand{\caption}[2][\relax]{\MYoriglatexcaption[#2]{#2}}
%\fi
% endfloat.sty was written by James Darrell McCauley, Jeff Goldberg and 
% Axel Sommerfeldt. This package may be useful when used in conjunction with 
% IEEEtran.cls'  captionsoff option. Some IEEE journals/societies require that
% submissions have lists of figures/tables at the end of the paper and that
% figures/tables without any captions are placed on a page by themselves at
% the end of the document. If needed, the draftcls IEEEtran class option or
% \CLASSINPUTbaselinestretch interface can be used to increase the line
% spacing as well. Be sure and use the nomarkers option of endfloat to
% prevent endfloat from "marking" where the figures would have been placed
% in the text. The two hack lines of code above are a slight modification of
% that suggested by in the endfloat docs (section 8.4.1) to ensure that
% the full captions always appear in the list of figures/tables - even if
% the user used the short optional argument of \caption[]{}.
% IEEE papers do not typically make use of \caption[]'s optional argument,
% so this should not be an issue. A similar trick can be used to disable
% captions of packages such as subfig.sty that lack options to turn off
% the subcaptions:
% For subfig.sty:
% \let\MYorigsubfloat\subfloat
% \renewcommand{\subfloat}[2][\relax]{\MYorigsubfloat[]{#2}}
% However, the above trick will not work if both optional arguments of
% the \subfloat command are used. Furthermore, there needs to be a
% description of each subfigure *somewhere* and endfloat does not add
% subfigure captions to its list of figures. Thus, the best approach is to
% avoid the use of subfigure captions (many IEEE journals avoid them anyway)
% and instead reference/explain all the subfigures within the main caption.
% The latest version of endfloat.sty and its documentation can obtained at:
% http://www.ctan.org/pkg/endfloat
%
% The IEEEtran \ifCLASSOPTIONcaptionsoff conditional can also be used
% later in the document, say, to conditionally put the References on a 
% page by themselves.




% *** PDF, URL AND HYPERLINK PACKAGES ***
%
\usepackage{url}
% url.sty was written by Donald Arseneau. It provides better support for
% handling and breaking URLs. url.sty is already installed on most LaTeX
% systems. The latest version and documentation can be obtained at:
% http://www.ctan.org/pkg/url
% Basically, \url{my_url_here}.




% *** Do not adjust lengths that control margins, column widths, etc. ***
% *** Do not use packages that alter fonts (such as pslatex).         ***
% There should be no need to do such things with IEEEtran.cls V1.6 and later.
% (Unless specifically asked to do so by the journal or conference you plan
% to submit to, of course. )


% correct bad hyphenation here
\hyphenation{op-tical net-works semi-conduc-tor}


\begin{document}
%
% paper title
% Titles are generally capitalized except for words such as a, an, and, as,
% at, but, by, for, in, nor, of, on, or, the, to and up, which are usually
% not capitalized unless they are the first or last word of the title.
% Linebreaks \\ can be used within to get better formatting as desired.
% Do not put math or special symbols in the title.
\title {An Implementation of a Portable Impedance\\Tomography Platform System}

%
%
% author names and IEEE memberships
% note positions of commas and nonbreaking spaces ( ~ ) LaTeX will not break
% a structure at a ~ so this keeps an author's name from being broken across
% two lines.
% use \thanks{} to gain access to the first footnote area
% a separate \thanks must be used for each paragraph as LaTeX2e's \thanks
% was not built to handle multiple paragraphs
%

\author{Ashton Johnson\thanks{Prof. Emil Jovanov with the Department of Electrical and Computer Engineering, University of Alabama in Huntsville, Huntsville, AL 35899}}% <-this % stops a space


% note the % following the last \IEEEmembership and also \thanks - 
% these prevent an unwanted space from occurring between the last author name
% and the end of the author line. i.e., if you had this:
% 
% \author{....lastname \thanks{...} \thanks{...} }
%                     ^------------^------------^----Do not want these spaces!
%
% a space would be appended to the last name and could cause every name on that
% line to be shifted left slightly. This is one of those "LaTeX things". For
% instance, "\textbf{A} \textbf{B}" will typeset as "A B" not "AB". To get
% "AB" then you have to do: "\textbf{A}\textbf{B}"
% \thanks is no different in this regard, so shield the last } of each \thanks
% that ends a line with a % and do not let a space in before the next \thanks.
% Spaces after \IEEEmembership other than the last one are OK (and needed) as
% you are supposed to have spaces between the names. For what it is worth,
% this is a minor point as most people would not even notice if the said evil
% space somehow managed to creep in.



% The paper headers
\markboth{CPE 621, FALL 2017}%
{Shell \MakeLowercase{\textit{Johnson}}: Portable Impedance Tomography Platform}
% The only time the second header will appear is for the odd numbered pages
% after the title page when using the twoside option.
% 
% *** Note that you probably will NOT want to include the author's ***
% *** name in the headers of peer review papers.                   ***
% You can use \ifCLASSOPTIONpeerreview for conditional compilation here if
% you desire.

% If you want to put a publisher's ID mark on the page you can do it like
% this:
%\IEEEpubid{0000--0000/00\$00.00~\copyright~2015 IEEE}
% Remember, if you use this you must call \IEEEpubidadjcol in the second
% column for its text to clear the IEEEpubid mark.

% use for special paper notices
%\IEEEspecialpapernotice{(Invited Paper)}

% make the title area
\maketitle

% As a general rule, do not put math, special symbols or citations
% in the abstract or keywords.
\begin{abstract}
Electrical impedance analysis is a non-invasive technique used for determining the density and composition of various materials. Using multiple electrode pairs, we developed a solution to provide two dimensional (2D) tomography imaging. A portable bio-impedance tomography unit is realized using the Analog Devices AD5933 integrated circuit for measuring complex impedance, an analog front end for interfacing to the human body and analog multiplexers. We implement this as a portable device that uses Bluetooth Low Energy as the interface.
\end{abstract}

% Note that keywords are not normally used for peerreview papers.
\begin{IEEEkeywords}
bioimpedance, AD5933, tomography, impedance, Bluetooth, BLE, smartphone, PSoC, Cypress, pmod
\end{IEEEkeywords}



% For peer review papers, you can put extra information on the cover
% page as needed:
% \ifCLASSOPTIONpeerreview
% \begin{center} \bfseries EDICS Category: 3-BBND \end{center}
% \fi
%
% For peerreview papers, this IEEEtran command inserts a page break and
% creates the second title. It will be ignored for other modes.
%\IEEEpeerreviewmaketitle



\section{Introduction}

\IEEEPARstart{T}{he} impedance of an object can be a useful classifier of an object and its composition. In this paper, we will focus on its use as a physiological measurement tool, known as bioelectical impedance analysis. This is accomplished using electrodes connected to a device that generates a stimulus a given frequency and measures the response. By increasing the number of electrodes and multiplexing them, it is possible to develop a two-dimensional (2D) representation of the object under test. This is known as Electrical Impedance Tomography (EIT). While this can be accomplished using laboratory test equipment and post-processing with a computer, medical uses dictate portability as a requirements for such systems where the it could be affixed to a person while still allowing for complete mobility. In this paper, we describe a portable system using an impedance analyzer integrated circuit. For proper interfacing to the human body, we implement a 4-electrode, analog front end. This uses two electrodes driven by a voltage controlled current source, and two electrodes for a differential voltage measurement capturing the response. We added analog multiplexers to allow for 36 different spatial measurements. Finally, we added a custom Bluetooth Low Energy (BLE) service for complete control of the platform. 
\newline 
% You must have at least 2 lines in the paragraph with the drop letter
% (should never be an issue)

\section{Literature Survey}
The first part of this paper discusses the basic concepts of impedance and how we can quantify it. We then discuss existing application of impedance as a tool.  We then discuss the challenges of measuring impedance of the human body. We then present our design of a portable impedance tomography platform. We go on to describe out results, and then discuss related works. 


%\hfill acj
%\hfill October 22, 2017

\section{Project Goals}
The goals of this project are to develop a portable solution for impedance measurements. It shall have the following features:
% needed in second column of first page if using \IEEEpubid
%\IEEEpubidadjcol
\begin{itemize}
\item{Time Domain Impedance Measurements}
\item{Frequency Domain Impedance Measurements}
\item{Two-dimensional Impedance Measurements}
\item{Effective Body Interface}
\item{Portability}
\itme{Bluetooth Connectivity}
\end{itemize}


\section{Impedance Overview}
Impedance can be defined as the effective resistance of an electric circuit or component to alternating current, arising from the combined effects of ohmic resistance and frequency dependent reactance. At a frequency of zero Hertz, the impedance is simply the ohmic resistance. The impedance of a circuit can be thought of as a complex number, where the real part is the ohmic resistance and the imaginary part is the reactance. Therefore the impedance can be expressed as \[Z=R+jX\]In application, there are two ideal components that that have purely reactive impedance. That is, the \(R\) is zero; therefore \(Z=jX\). One component is an inductor, whose impedance is expressed as \[Z_L=j\omegaL\] where \(\omega=2*\pi*freq(Hz)\) and L is the inductance in Henries. The other component is a capacitor, whose impedance is expressed as \[Z_C=\frac{1}{j\omegaC}\] where \(\omega=2*\pi*freq(Hz)\) and C is the capacitance in Farads.\newline

\subsection{Applications}
There are numerous applications where impedance can be used as an investigative tool. Electrochemical Impedance Spectroscopy is a process that is used to characterize chemical coatings on materials\cite{macdonald_reflections_2006}. Practically, this allows for analysis such as the anodic corrosion of iron in sulfuric acid, the effectiveness of sensitizers in dye-sensitized solar cells\cite{wang_electrochemical_2005}, or monitoring the corrosion of reinforced concrete\cite{ribeiro_use_2015}. \cite{barsoukov_impedance_2005} focuses a great deal on using Impedance Spectroscopy to analyze battery chemical compositions. Impedance Spectroscopy is also widely used on organic matter, specifically on the human body. This is more formally defined as Electrical Bioimpedance Spectroscopy (BIS), or Bioelectrical Impedance Analysis (BIA). This includes the more commonly known Body Composition Measurements such as total body water (TBW) content and body fat estimation. These devices widely spread in the market targeting personal health monitoring devices. These implementations include hand grip style analyzers\cite{noauthor_amazon.com:_nodate} or are included in floor scales\cite{noauthor_amazon.com:_nodate-1}.
\newline
The human body can be generalized in a circuit that has a complex impedance. It is a combination of a purely resistive element, as well as capacitive elements. The exact values of these components are dependent on many factors including electrode placement, water content, salt content, movement and other factors.\cite{lukaski_assessment_1985} performs a thorough characterization of the human body. Impedance Pneuemography has been used for monitoring an individuals respiration rates and cardiac output\cite{grenvik_impedance_1972},\cite{larsen_impedance_1984},\cite{ernst_impedance_1999}.
\newline
This paper focuses specifically with two-dimensional (2D) and three-dimensional (3D) applications of impedance measurement. This is known at Electrical Impedance Tomography (EIT), and is an imaging technique already in use\cite{noauthor_body_nodate},\cite{noauthor_our_nodate}, but not as widely as the simpler body composition analyzers. This imaging can be performed at a single frequency or across a frequency spectrum. It's worth noting here that EIU is a not a replacement for Magnetic Resonance Imaging (MRI) or X-ray computerized tomography (CT) Scans. This is because unlike x-ray vector fields in CT scan, the current flowing across the electrodes of a EIT scanner is not fixed in the transverse plane of the measurement electrodes. The results of the EIT scan are better if the material composition is uniform as you extend tangentially away from the transverse scan plane. An example of this is the thorax when considering a scan of the lungs, as the lungs are more than an order of magnitude taller than the height of the electrodes place around the thorax. Unlike CT scans, EIT does not involve ionizing radiation, and also has a much higher temporal resolution in the range of 0.1 millisecond\cite{noauthor_electrical_2004}. \cite{adler_electrical_2017} thoroughly covers the considerations of EIT. Fielded applications include continuous respiration monitoring and breast cancer detection\cite{assenheimer_t-scan_2001}. Ongoing areas of research with promising results are in cervical screenings\cite{brown_relation_2000} but are still faced with challenges when used a method of detecting neural activity\cite{gilad_impedance_2009}.

\subsection{Measuring Impedance}
The measurement of an unknown impedance can be found by measuring the voltage or current waveforms at two different points in the circuit, shown in Fig~\ref{fig:zun}\cite{noauthor_oscilloscope_nodate}. The magnitude and angle of the impedance of an unknown object at a given frequency can be expressed as

\begin{figure} %reference https://en.wikibooks.org/wiki/LaTeX/Floats,_Figures_and_Captions for [!h] options on placing the figures
\centering
\includegraphics[width=5cm ]{./graphics/zun.png}
\centering
\caption{Circuit for measuring unknown impedance}
\label{fig:zun}
\end{figure}


\[Z_{uut}=\frac{V_{A2}R_{ref}}{\sqrt{V^2_{A1}-2V_{A1}V_{A2}cos\theta+V^2_{A2}}}\]
\newline
\[\alpha_{uut}=\theta-tan^{-1}\frac{-V_{A2}sin\theta}{V_{A1}-V_{A2}cos\theta} \]
where \(\theta\) is the phase difference between measured voltage at A2 relative to A1.
\newline
Another method of determining the impedance is performing a discrete Fourier transform (DFT) on the sampled data of a sinusoidal wave passed though the circuit. This is defined as
\[X(f)=\sum_{n=0}^{N}(x(n)(cos(n)-jsin(n)))\]
where \(X(f)\) is expressed complex number \(R+jX\), \(x(n)\)is the resultant waveform at the output of the unknown impedance network. \(cos(n)\) and \(sin(n)\) are the reference waveform being passed into the unknown impedance. The magnitude is defined as \(\sqrt{R^2+X^2}\) and the phase is defined as \(tan^{-1}(Z/R)\times\frac{180\textdegree}{\pi}\) in degrees\cite{noauthor_ad5933_nodate}.

\section{Hardware Implementation}

\subsection{AD5933 Impedance Analyzer}
Analog Devices has introduced the AD5933 integrated circuit (IC) as a monolithic platform for impedance measurement. It is a 1 Megasample per second, 12-Bit Impedance Converter\cite{noauthor_ad5933_nodate}. The response signal from the impedance is sampled by the on-board ADC and a DFT is processed by an on-board Digital Signal Processing (DSP) engine. The DFT algorithm returns a real (R) and an imaginary (I) result at each output frequency. Once calibrated, the relative magnitude and phase of the impedance at each frequency point along the sweep is easily calculated. This is done off chip using the real and imaginary register contents, which is accessed from over an Inter-Integrated Circuit (I2C) interface.\newline
\subsection{Analog Front End}
The AD5933 has been evaluated as a suitable platform for
BIA\cite{breniuc_wearable_2014},\cite{ferreira_ad5933-based_2011},\cite{harder_smart_2016},\cite{bakr_aging_2016},\cite{pliquett_interfacing_2012}, but the analog interface to the unit-under-test (UUT) is not design specific; is a simple voltage digital-to-analog converter (DAC) that drives a single-ended voltage-mode analog-to-digital converter (ADC). For bioimpedance applications, it is important to have a constant current amplitude over the frequency range of interest to ensure accurate measurements\cite{bertemes-filho_mirrored_2012},\cite{seoane_analog_2008},\cite{pliquett_interfacing_2012},\cite{bera_battery-based_2013}. Additionally, it is also important to not exceed safe current levels passing through the human body as defined in IEC60601-1\cite{noauthor_iec_nodate}. This limit is defined as the Patient Auxiliary Current. For DC current, this is limited to 0.05 miliAmperes. For AC current, this is limited to 0.5 miliAmperes. Using a controlled current source, we can precisely control the current flowing through the test subject. 



\begin{figure} %reference https://en.wikibooks.org/wiki/LaTeX/Floats,_Figures_and_Captions for [!h] options on placing the figures
\centering
\includegraphics[width=9cm ]{./graphics/front_end.png}
\centering
\caption{Analog Front End}
\label{fig:front_end}
\end{figure}


To accommodate the aforementioned limitations of the AD5933, we implemented a voltage-controlled current source for providing the stimulus the UUT. We also provided a differential input so that the resulting voltage is measured between two electrodes. Differential measurements should eliminate any common mode interference present on the electrodes. The analog front end is based the design described in \cite{harder_smart_2016}. The design shown in Fig~\ref{fig:front_end} uses an Analog Devices AD8130 differential receiver amplifier\cite{noauthor_ad8130_nodate} as a wide-band voltage controlled current source.

\subsubsection{Electrode Multiplexing}
The AD5933 Impedance analyzer using an current mode analog front end allows for taking measurements across four electrodes; two for the excitation and two for the sensing. Spatially, this allows only for a one-dimensional (1D) measurement. For tomography applications, driving one pair of electrodes and sensing on multiple pairs of electrodes is required. \cite{hua_using_1993},\cite{dimas_development_2017},\cite{wang_electrical_nodate}, \cite{vilchez-monge_image_2017} have demonstrated implementations. This allows for a two-dimensional (2D) dataset. For a set of N electrodes, in our case 4 sets, the number of measurements we can have is defined as \[Measurement_{Combinations}=\sum_{i=1}^{i=N-1}i\]. Therefore for 4 sets of sense electrodes, we will have 6 different measurements. This demonstrated by the number of vectors in Fig~\ref{fig:combo4}

\begin{figure} %reference https://en.wikibooks.org/wiki/LaTeX/Floats,_Figures_and_Captions for [!h] options on placing the figures
\centering
\includegraphics[width=5cm ]{./graphics/combo4.png}
\centering
\caption{Electrode Combinations}
\label{fig:combo4}
\end{figure}

For this design, we not only multiplexed the voltage sense electrodes, but also the current drive electrodes. This means we have \[\sum_{i=1}^{i=N-1}i\times\sum_{i=1}^{i=N-1}i=6\times6=36\]
different combinations of measurements we can collect. The multiplexing is realized using two Analog Devices AD1209 analog multiplexers. This IC is a differential 4-to-1 multiplexer. This allows us to switch both the drive and the sense. A shortcoming of this design is that both addresses are switching using the same address bus, this limits the independent switching control of the positive and negative electrode pairs. This limits the to back down to the original six.

\begin{figure} %reference https://en.wikibooks.org/wiki/LaTeX/Floats,_Figures_and_Captions for [!h] options on placing the figures
\centering
\includegraphics[width=5cm ]{./graphics/mux.png}
\centering
\caption{Electrode Mux Configuration}
\label{fig:mux}
\end{figure}


\subsection{Device Controller}
The Cypress CY5671\cite{noauthor_cy5671:_nodate} module was chosen for as the localized controller on the device. This module is provides integrate I2C and Bluetooth Low Energy (BLE) peripherals for communication between the AD5933 and user application respectively. 

\subsection{Printed Circuit Board Assembly}
A custom printed circuit board assembly in Fig~\ref{fig:pcb} was developed for hosting the AD5933 and AFE as well as providing an easy electro-mechanical interface to the CY5671. This board was developed using KiCad open source EDA software\cite{noauthor_kicad_nodate}. This PCBA was realized using four layers with priority routing given to the analog signals.

\begin{figure} %reference https://en.wikibooks.org/wiki/LaTeX/Floats,_Figures_and_Captions for [!h] options on placing the figures
\centering
\includegraphics[width=5cm ]{./graphics/pcba.png}
\centering
\caption{Custom PCBA}
\label{fig:pcb}
\end{figure}


\section{Software Implementation}
\subsection{Device States}
TBD
\subsection{Bluetooth Communication}

Bluetooth Low Energy (BLE) was chosen as the communication method to the platform. A wireless method provides for user conveinience as tethering is not required, but also provides for isolation from noise generating computers and other equipment. There a numerous wireless protocols that can be used for interfacing to microcontrollers, but given the prevalence of BLE capable smartphones, this seems to be the obvious choice. A custom Bluetooth Low Energy service was implemented on the PSoC 4 BLE device. The Bluetooth service is composed of multiple characteristics. There are a reflection of all of the available registers on the AD5933. The result register includes the real and imaginary result from the AD5933, but also the frequency, transmit and receive multiplexer state and current time. 

\begin{figure} %reference https://en.wikibooks.org/wiki/LaTeX/Floats,_Figures_and_Captions for [!h] options on placing the figures
\centering
\includegraphics[width=5cm ]{../graphics/ble_service.png}
\centering
\caption{BLE Service}
\label{fig:ble}
\end{figure}



\subsection{Android Application}
TBD


\section{Results}
\subsection{Single Frequency}
TBD
\subsection{Calibration}
TBD
\subsection{Frequency Sweep}
TBD


\section {Related Work}



% \tt{
% class DefaultFPGARoccConfig \newline
% class DefaultFPGA32KBL2Config \newline
% class DefaultFPGA256KBL2Config \newline
% class DefaultFPGADualCoreConfig}\rm{}



% An example of a floating figure using the graphicx package.
% Note that \label must occur AFTER (or within) \caption.
% For figures, \caption should occur after the \includegraphics.
% Note that IEEEtran v1.7 and later has special internal code that
% is designed to preserve the operation of \label within \caption
% even when the captionsoff option is in effect. However, because
% of issues like this, it may be the safest practice to put all your
% \label just after \caption rather than within \caption{}.
%
% Reminder: the "draftcls" or "draftclsnofoot", not "draft", class
% option should be used if it is desired that the figures are to be
% displayed while in draft mode.
%
%\begin{figure}[!t]
%\centering
%\includegraphics[width=2.5in]{myfigure}
% where an .eps filename suffix will be assumed under latex, 
% and a .pdf suffix will be assumed for pdflatex; or what has been declared
% via \DeclareGraphicsExtensions.
%\caption{Simulation results for the network.}
%\label{fig_sim}
%\end{figure}

% Note that the IEEE typically puts floats only at the top, even when this
% results in a large percentage of a column being occupied by floats.

% An example of a double column floating figure using two subfigures.
% (The subfig.sty package must be loaded for this to work.)
% The subfigure \label commands are set within each subfloat command,
% and the \label for the overall figure must come after \caption.
% \hfil is used as a separator to get equal spacing.
% Watch out that the combined width of all the subfigures on a 
% line do not exceed the text width or a line break will occur.
%
%\begin{figure*}[!t]
%\centering
%\subfloat[Case I]{\includegraphics[width=2.5in]{box}%
%\label{fig_first_case}}
%\hfil
%\subfloat[Case II]{\includegraphics[width=2.5in]{box}%
%\label{fig_second_case}}
%\caption{Simulation results for the network.}
%\label{fig_sim}
%\end{figure*}
%
% Note that often IEEE papers with subfigures do not employ subfigure
% captions (using the optional argument to \subfloat[]), but instead will
% reference/describe all of them (a), (b), etc., within the main caption.
% Be aware that for subfig.sty to generate the (a), (b), etc., subfigure
% labels, the optional argument to \subfloat must be present. If a
% subcaption is not desired, just leave its contents blank,
% e.g., \subfloat[].

% An example of a floating table. Note that, for IEEE style tables, the
% \caption command should come BEFORE the table and, given that table
% captions serve much like titles, are usually capitalized except for words
% such as a, an, and, as, at, but, by, for, in, nor, of, on, or, the, to
% and up, which are usually not capitalized unless they are the first or
% last word of the caption. Table text will default to \footnotesize as
% the IEEE normally uses this smaller font for tables.
% The \label must come after \caption as always.
%
%\begin{table}[!t]
%% increase table row spacing, adjust to taste
%\renewcommand{\arraystretch}{1.3}
% if using array.sty, it might be a good idea to tweak the value of
% \extrarowheight as needed to properly center the text within the cells
%\caption{An Example of a Table}
%\label{table_example}
%\centering
%% Some packages, such as MDW tools, offer better commands for making tables
%% than the plain LaTeX2e tabular which is used here.
%\begin{tabular}{|c||c|}
%\hline
%One & Two\\
%\hline
%Three & Four\\
%\hline
%\end{tabular}
%\end{table}

% Note that the IEEE does not put floats in the very first column
% - or typically anywhere on the first page for that matter. Also,
% in-text middle ("here") positioning is typically not used, but it
% is allowed and encouraged for Computer Society conferences (but
% not Computer Society journals). Most IEEE journals/conferences use
% top floats exclusively. 
% Note that, LaTeX2e, unlike IEEE journals/conferences, places
% footnotes above bottom floats. This can be corrected via the
% \fnbelowfloat command of the stfloats package.

\section{Conclusion}
TBD

% if have a single appendix:
%\appendix[Proof of the Zonklar Equations]
% or
%\appendix  % for no appendix heading
% do not use \section anymore after \appendix, only \section*
% is possibly needed

% use appendices with more than one appendix
% then use \section to start each appendix
% you must declare a \section before using any
% \subsection or using \label (\appendices by itself
% starts a section numbered zero.)
%



%    \appendix{Source Material}
%    All source material for this project is provided under the MIT license and is available at \url{https://github.com/ashtonchase/portable_impedance_tomography} . 

% use section* for acknowledgment

%     \section*{Acknowledgment}
%     TBD


% Can use something like this to put references on a page
% by themselves when using endfloat and the captionsoff option.
\ifCLASSOPTIONcaptionsoff
  \newpage
\fi



% trigger a \newpage just before the given reference
% number - used to balance the columns on the last page
% adjust value as needed - may need to be readjusted if

% the document is modified later
%\IEEEtriggeratref{8}
% The "triggered" command can be changed if desired:
%\IEEEtriggercmd{\enlargethispage{-5in}}

% references section

% can use a bibliography generated by BibTeX as a .bbl file
% BibTeX documentation can be easily obtained at:
% http://mirror.ctan.org/biblio/bibtex/contrib/doc/
% The IEEEtran BibTeX style support page is at:
% http://www.michaelshell.org/tex/ieeetran/bibtex/
\bibliographystyle{IEEEtran}
% argument is your BibTeX string definitions and bibliography database(s)
\bibliography{report}
%
% <OR> manually copy in the resultant .bbl file
% set second argument of \begin to the number of references
% (used to reserve space for the reference number labels box)
%\begin{thebibliography}{1}
 
%\bibitem{Asanović:EECS-2016-17}
 
%\end{thebibliography}

% biography section
% 
% If you have an EPS/PDF photo (graphicx package needed) extra braces are
% needed around the contents of the optional argument to biography to prevent
% the LaTeX parser from getting confused when it sees the complicated
% \includegraphics command within an optional argument. (You could create
% your own custom macro containing the \includegraphics command to make things
% simpler here.)
%\begin{IEEEbiography}[{\includegraphics[width=1in,height=1.25in,clip,keepaspectratio]{mshell}}]{Michael Shell}
% or if you just want to reserve a space for a photo:

% if you will not have a photo at all:
\begin{IEEEbiographynophoto}{Ashton Johnson}
 received his B.E. degree in wireless engineering from Auburn University in 2011. He is currently seeking his Master's degree in computer engineering at the University of Alabama in Huntsville. He current works as an electrical engineer at Dynetics, Inc. in Huntsville. 
\end{IEEEbiographynophoto}

% insert where needed to balance the two columns on the last page with
% biographies
%\newpage

% You can push biographies down or up by placing
% a \vfill before or after them. The appropriate
% use of \vfill depends on what kind of text is
% on the last page and whether or not the columns
% are being equalized.

%\vfill

% Can be used to pull up biographies so that the bottom of the last one
% is flush with the other column.
%\enlargethispage{-5in}



% that's all folks
\end{document}



